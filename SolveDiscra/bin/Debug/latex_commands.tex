\documentclass[a4paper,10pt]{report} % формат бумаги А4, шрифт по умолчанию - 12pt

% заметь, что в квадратных скобках вводятся необязательные аргументы пакетов.
% а в фигурных - обязательные

\usepackage[T2A]{fontenc} % поддержка кириллицы в Латехе
\usepackage[utf8]{inputenc} % включаю кодировку ютф8
\usepackage[english,russian]{babel} % использую русский и английский языки с переносами

\usepackage{indentfirst} % делать отступ в начале параграфа
\usepackage{amsmath} % математические штуковины
\usepackage{mathtools} % еще математические штуковины
\usepackage{mathtext}
\usepackage{multicol} % подключаю мультиколоночность в тексте
\usepackage{graphicx} % пакет для вставки графики, я хз нахуя он нужен в этом документе
\usepackage{listings} % пакет для вставки кода
\usepackage[table]{xcolor}% http://ctan.org/pkg/xcolor for coloring the inside of a cell
\usepackage[lofdepth,lotdepth]{subfig} % so we can place figures side by side

\usepackage{geometry} % меняю поля страницы
\usepackage{caption}

%из параметров ниже понятно, какие части полей страницы меняются:
\geometry{left=1.4cm}
\geometry{right=2cm}
\geometry{top=1.5cm}
\geometry{bottom=2cm}

\renewcommand{\baselinestretch}{1} % меняю ширину между строками на 1.5
\righthyphenmin=2
\begin{document}


\begin{titlepage}
\newpage

\begin{center}
{\large НАЦИОНАЛЬНЫЙ ИССЛЕДОВАТЕЛЬСКИЙ УНИВЕРСИТЕТ \\
«ВЫСШАЯ ШКОЛА ЭКОНОМИКИ» 							\\
Дисциплина: «Дискретная математика»}

\vfill % заполняет длину страницы вертикально

{\large Домашнее задание 1}

\bigskip

%\underline{ Исследование алгоритмов решения \\ метрической задачи коммивояжера \\ }
Вариант 002\\

\vfill

\begin{flushright}
Выполнил: Абрамов Артем,\\
студент группы БПИ1511\medskip \\
Преподаватель: Авдошин С.М., \\
профессор департамента \\
программной инженерии \\
факультета компьютерных наук
\end{flushright}

\vfill

Москва \number\year

\end{center}
\end{titlepage}




1. \quad На плоскости задано множество точек $V = {1, 2, 3, 4, 5, 6}$ своими координатами
$(x = 4,y = 1),(x = 4,y = 3),(x = 2,y = 7),(x = 9,y = 6),(x = 10,y = 7),(x = 6,y = 10)$ \\

\smallskip

2. \quad Вычислим элементы $d_{ij}$ весовой матрицы смежности графа 
$G = <V, V*V>$ по формуле $dij = |xi - xj|  +  |yi - yj|$

\begin{flushleft}\begin{tabular}[]{|c|c|c|c|c|c|c|}
\hline
     0  &      2 &      8 &      10 &     12 &      11\\
\hline
      2 &       0 &      6 &      8 &      10 &      9\\
\hline
      8 &      6 &     0  &      8 &      8  &      7\\
\hline
      10 &      8 &      8 &     0  &      2 &      7\\
\hline
      12 &      10 &     8 &     2 &    0 &      7\\
\hline
      11 &      9 &      7 &      7 &      7 &    0\\
\hline
\end{tabular}
\end{flushleft}

\smallskip

3. \quad Используя метод ветвей и границ, найдем множество кодов всех оптимальных гамильтоновых
 циклов являющихся решением задачи коммивояжера на графе G.
Петли не могут быть частью решения. Поэтому положим диагональные элементы равными бесконечности.

\begin{flushleft}\begin{tabular}[]{|c|c|c|c|c|c|c|}
\hline
$\infty$&      2 &      8 &      10 &      12 &      11\\
\hline
      2 & $\infty$ &      6 &      8 &      10 &      9\\
\hline
      8 &      6 & $\infty$ &      8 &      8 &      7\\
\hline
      10 &      8 &      8 & $\infty$ &      2 &      7\\
\hline
      12 &      10 &      8 &      2 & $\infty$ &      7\\
\hline
      11 &      9 &      7 &      7 &      7 & $\infty$\\
\hline
\end{tabular}
\end{flushleft}

Обозначим через
$ S = \{ p:V \rightarrow V \} (p(l) = l) \& ( \forall j \subset V ) ( \forall j \subset V) ((p(i) = p(j)) \Longrightarrow (i = j))$
множество кодов всех гамильтоновых циклов $v=(p_1,p_2,p_3,p_4,p_5,p_6,p_1)$ графа G, 
заданного весовой матрицей смежности D. 
Здесь $p_{i}$ используется в качестве сокращенной записи $p(i)$.
Найдем нижнюю границу $b$ множества $S$


% make the captions stick to the LEFT of the page
\captionsetup{justification=raggedright,
singlelinecheck=false
}

\captionsetup[subfloat]{labelformat=empty}


\begin{table}[ht]
\subfloat[][]{\begin{tabular}[]{|c|c|c|c|c|c|c|c|c|}
\hline
s & 1 & 2 & 3 & 4 & 5 & 6 & min $\alpha$ \\
\hline
1     & $\infty$ &      2 &      8 &      10 &      12 &      11  & 2\\
\hline
2     &       2 & $\infty$ &      6 &      8 &      10 &      9    &  2\\
\hline
3     &       8 &      6 & $\infty$ &      8 &      8 &      7     &  6 \\
\hline
4     &       10 &      8 &      8 & $\infty$ &      2 &      7    & 2\\
\hline
5     &       12 &      10 &      8 &      2 & $\infty$ &      7  & 2\\
\hline
6     &       11 &      9 &      7 &      7 &      7 & $\infty$   & 7 \\
\hline
\end{tabular}
}\hfill
\subfloat[][]{\begin{tabular}[]{|c|c|c|c|c|c|c|c|}
\hline
s & 1 & 2 & 3 & 4 & 5 & 6 \\
\hline
1     & $\infty$ &      0 &      6 &      8 &      10 &      9 \\
\hline
2     &       0 & $\infty$ &      4 &      6 &      8 &      7  \\
\hline
3     &       2 &      0 & $\infty$ &      2 &      2 &      1   \\
\hline
4     &       8 &      6 &      6 & $\infty$ &      0 &      5 \\
\hline
5     &       10 &      8 &      6 &      0 & $\infty$ &      5 \\
\hline
6     &       4 &      2 &      0 &      0 &      0 & $\infty$ \\
\hline
min $\beta$ &  0 &      0 &      0 &      0 &      0 &     1 \\
\hline
\end{tabular}
}\captionof*{table}{$b = \alpha + 	\beta = 28$}
\end{table}

\newpage

Определим дугу ветвления для разбиения множества  s\\
\begin{flushleft}\begin{tabular}[]{|c|c|c|c|c|c|c|}
\hline
s & 1 & 2 & 3 & 4 & 5 & 6\\
\hline
1 & $\infty$ &      0 &      6 &      5 &      3 &      2\\
\hline
2 &      0 & $\infty$ &      5 &      4 &      0 &      1\\
\hline
3 &      6 &      5 & $\infty$ &      0 &      0 &      1\\
\hline
4 &      5 &      4 &      0 & $\infty$ &      3 &      0\\
\hline
5 &      3 &      0 &      0 &      3 & $\infty$ &      2\\
\hline
6 &      2 &      1 &      1 &      0 &      2 & $\infty$\\
\hline
\end{tabular}
\captionof*{table}{(1,2)}
\end{flushleft}


\begin{table}[ht]
\subfloat[][]{\begin{tabular}[]{|c|c|c|c|c|c|c|c}
\hline
s0 & 1 & 2 & 3 & 4 & 5 & 6 & min\\
\hline
1 & $\infty$ & \cellcolor{yellow}$\infty$ &      6 &      5 &      3 &      2 & 2\\
\hline
2 &      0 & $\infty$ &      5 &      4 &      0 &      1 & 0\\
\hline
3 &      6 &      5 & $\infty$ &      0 &      0 &      1 & 0\\
\hline
4 &      5 &      4 &      0 & $\infty$ &      3 &      0 & 0\\
\hline
5 &      3 &      0 &      0 &      3 & $\infty$ &      2 & 0\\
\hline
6 &      2 &      1 &      1 &      0 &      2 & $\infty$ & 0\\
\hline
min &      0 &      0 &      0 &      0 &      0 &      0\\
\end{tabular}
}\hfill
\subfloat[][]{\begin{tabular}[]{|c|c|c|c|c|c|c|}
\hline
s0 & 1 & 2 & 3 & 4 & 5 & 6\\
\hline
1 & $\infty$ & $\infty$ &      4 &      3 &      1 &      0\\
\hline
2 &      0 & $\infty$ &      5 &      4 &      0 &      1\\
\hline
3 &      6 &      5 & $\infty$ &      0 &      0 &      1\\
\hline
4 &      5 &      4 &      0 & $\infty$ &      3 &      0\\
\hline
5 &      3 &      0 &      0 &      3 & $\infty$ &      2\\
\hline
6 &      2 &      1 &      1 &      0 &      2 & $\infty$\\
\hline
\end{tabular}
}\captionof*{table}{b0 = b + 2 + 0 = 30}
\end{table}


\begin{table}[ht]
\subfloat[][]{\begin{tabular}[]{|c|c|c|c|c|c|c}
\hline
s1 & 1 & 3 & 4 & 5 & 6 & min\\
\hline
2 & \cellcolor{yellow}$\infty$ &      5 &      4 &      0 &      1 & 0\\
\hline
3 &      6 & $\infty$ &      0 &      0 &      1 & 0\\
\hline
4 &      5 &      0 & $\infty$ &      3 &      0 & 0\\
\hline
5 &      3 &      0 &      3 & $\infty$ &      2 & 0\\
\hline
6 &      2 &      1 &      0 &      2 & $\infty$ & 0\\
\hline
min &      2 &      0 &      0 &      0 &      0\\
\end{tabular}
}\hfill
\subfloat[][]{\begin{tabular}[]{|c|c|c|c|c|c|}
\hline
s1 & 1 & 3 & 4 & 5 & 6\\
\hline
2 & $\infty$ &      5 &      4 &      0 &      1\\
\hline
3 &      4 & $\infty$ &      0 &      0 &      1\\
\hline
4 &      3 &      0 & $\infty$ &      3 &      0\\
\hline
5 &      1 &      0 &      3 & $\infty$ &      2\\
\hline
6 &      0 &      1 &      0 &      2 & $\infty$\\
\hline
\end{tabular}
}\captionof*{table}{b1 = b + 0 + 2 = 30}
\end{table}

\newpage

Определим дугу ветвления для разбиения множества  s1\\
\begin{flushleft}\begin{tabular}[]{|c|c|c|c|c|c|}
\hline
s1 & 1 & 3 & 4 & 5 & 6\\
\hline
2 & $\infty$ &      5 &      4 &      0 &      1\\
\hline
3 &      4 & $\infty$ &      0 &      0 &      1\\
\hline
4 &      3 &      0 & $\infty$ &      3 &      0\\
\hline
5 &      1 &      0 &      3 & $\infty$ &      2\\
\hline
6 &      0 &      1 &      0 &      2 & $\infty$\\
\hline
\end{tabular}
\captionof*{table}{(2,5)}
\end{flushleft}


\begin{table}[ht]
\subfloat[][]{\begin{tabular}[]{|c|c|c|c|c|c|c}
\hline
s10 & 1 & 3 & 4 & 5 & 6 & min\\
\hline
2 & $\infty$ &      5 &      4 & \cellcolor{yellow}$\infty$ &      1 & 1\\
\hline
3 &      4 & $\infty$ &      0 &      0 &      1 & 0\\
\hline
4 &      3 &      0 & $\infty$ &      3 &      0 & 0\\
\hline
5 &      1 &      0 &      3 & $\infty$ &      2 & 0\\
\hline
6 &      0 &      1 &      0 &      2 & $\infty$ & 0\\
\hline
min &      0 &      0 &      0 &      0 &      0\\
\end{tabular}
}\hfill
\subfloat[][]{\begin{tabular}[]{|c|c|c|c|c|c|}
\hline
s10 & 1 & 3 & 4 & 5 & 6\\
\hline
2 & $\infty$ &      4 &      3 & $\infty$ &      0\\
\hline
3 &      4 & $\infty$ &      0 &      0 &      1\\
\hline
4 &      3 &      0 & $\infty$ &      3 &      0\\
\hline
5 &      1 &      0 &      3 & $\infty$ &      2\\
\hline
6 &      0 &      1 &      0 &      2 & $\infty$\\
\hline
\end{tabular}
}\captionof*{table}{b10 = b1 + 1 + 0 = 31}
\end{table}


\begin{table}[ht]
\subfloat[][]{\begin{tabular}[]{|c|c|c|c|c|c}
\hline
s11 & 1 & 3 & 4 & 6 & min\\
\hline
3 &      4 & $\infty$ &      0 &      1 & 0\\
\hline
4 &      3 &      0 & $\infty$ &      0 & 0\\
\hline
5 & \cellcolor{yellow}$\infty$ &      0 &      3 &      2 & 0\\
\hline
6 &      0 &      1 &      0 & $\infty$ & 0\\
\hline
min &      0 &      0 &      0 &      0\\
\end{tabular}
}\hfill
\subfloat[][]{\begin{tabular}[]{|c|c|c|c|c|}
\hline
s11 & 1 & 3 & 4 & 6\\
\hline
3 &      4 & $\infty$ &      0 &      1\\
\hline
4 &      3 &      0 & $\infty$ &      0\\
\hline
5 & $\infty$ &      0 &      3 &      2\\
\hline
6 &      0 &      1 &      0 & $\infty$\\
\hline
\end{tabular}
}\captionof*{table}{b11 = b1 + 0 + 0 = 30}
\end{table}

\newpage

Определим дугу ветвления для разбиения множества  s0\\
\begin{flushleft}\begin{tabular}[]{|c|c|c|c|c|c|c|}
\hline
s0 & 1 & 2 & 3 & 4 & 5 & 6\\
\hline
1 & $\infty$ & $\infty$ &      4 &      3 &      1 &      0\\
\hline
2 &      0 & $\infty$ &      5 &      4 &      0 &      1\\
\hline
3 &      6 &      5 & $\infty$ &      0 &      0 &      1\\
\hline
4 &      5 &      4 &      0 & $\infty$ &      3 &      0\\
\hline
5 &      3 &      0 &      0 &      3 & $\infty$ &      2\\
\hline
6 &      2 &      1 &      1 &      0 &      2 & $\infty$\\
\hline
\end{tabular}
\captionof*{table}{(2,1)}
\end{flushleft}


\begin{table}[ht]
\subfloat[][]{\begin{tabular}[]{|c|c|c|c|c|c|c|c}
\hline
s00 & 1 & 2 & 3 & 4 & 5 & 6 & min\\
\hline
1 & $\infty$ & $\infty$ &      4 &      3 &      1 &      0 & 0\\
\hline
2 & \cellcolor{yellow}$\infty$ & $\infty$ &      5 &      4 &      0 &      1 & 0\\
\hline
3 &      6 &      5 & $\infty$ &      0 &      0 &      1 & 0\\
\hline
4 &      5 &      4 &      0 & $\infty$ &      3 &      0 & 0\\
\hline
5 &      3 &      0 &      0 &      3 & $\infty$ &      2 & 0\\
\hline
6 &      2 &      1 &      1 &      0 &      2 & $\infty$ & 0\\
\hline
min &      2 &      0 &      0 &      0 &      0 &      0\\
\end{tabular}
}\hfill
\subfloat[][]{\begin{tabular}[]{|c|c|c|c|c|c|c|}
\hline
s00 & 1 & 2 & 3 & 4 & 5 & 6\\
\hline
1 & $\infty$ & $\infty$ &      4 &      3 &      1 &      0\\
\hline
2 & $\infty$ & $\infty$ &      5 &      4 &      0 &      1\\
\hline
3 &      4 &      5 & $\infty$ &      0 &      0 &      1\\
\hline
4 &      3 &      4 &      0 & $\infty$ &      3 &      0\\
\hline
5 &      1 &      0 &      0 &      3 & $\infty$ &      2\\
\hline
6 &      0 &      1 &      1 &      0 &      2 & $\infty$\\
\hline
\end{tabular}
}\captionof*{table}{b00 = b0 + 0 + 2 = 32}
\end{table}


\begin{table}[ht]
\subfloat[][]{\begin{tabular}[]{|c|c|c|c|c|c|c}
\hline
s01 & 2 & 3 & 4 & 5 & 6 & min\\
\hline
1 & $\infty$ &      4 &      3 &      1 &      0 & 0\\
\hline
3 &      5 & $\infty$ &      0 &      0 &      1 & 0\\
\hline
4 &      4 &      0 & $\infty$ &      3 &      0 & 0\\
\hline
5 &      0 &      0 &      3 & $\infty$ &      2 & 0\\
\hline
6 &      1 &      1 &      0 &      2 & $\infty$ & 0\\
\hline
min &      0 &      0 &      0 &      0 &      0\\
\end{tabular}
}\hfill
\subfloat[][]{\begin{tabular}[]{|c|c|c|c|c|c|}
\hline
s01 & 2 & 3 & 4 & 5 & 6\\
\hline
1 & $\infty$ &      4 &      3 &      1 &      0\\
\hline
3 &      5 & $\infty$ &      0 &      0 &      1\\
\hline
4 &      4 &      0 & $\infty$ &      3 &      0\\
\hline
5 &      0 &      0 &      3 & $\infty$ &      2\\
\hline
6 &      1 &      1 &      0 &      2 & $\infty$\\
\hline
\end{tabular}
}\captionof*{table}{b01 = b0 + 0 + 0 = 30}
\end{table}

\newpage

Определим дугу ветвления для разбиения множества  s11\\
\begin{flushleft}\begin{tabular}[]{|c|c|c|c|c|}
\hline
s11 & 1 & 3 & 4 & 6\\
\hline
3 &      4 & $\infty$ &      0 &      1\\
\hline
4 &      3 &      0 & $\infty$ &      0\\
\hline
5 & $\infty$ &      0 &      3 &      2\\
\hline
6 &      0 &      1 &      0 & $\infty$\\
\hline
\end{tabular}
\captionof*{table}{(6,1)}
\end{flushleft}


\begin{table}[ht]
\subfloat[][]{\begin{tabular}[]{|c|c|c|c|c|c}
\hline
s110 & 1 & 3 & 4 & 6 & min\\
\hline
3 &      4 & $\infty$ &      0 &      1 & 0\\
\hline
4 &      3 &      0 & $\infty$ &      0 & 0\\
\hline
5 & $\infty$ &      0 &      3 &      2 & 0\\
\hline
6 & \cellcolor{yellow}$\infty$ &      1 &      0 & $\infty$ & 0\\
\hline
min &      3 &      0 &      0 &      0\\
\end{tabular}
}\hfill
\subfloat[][]{\begin{tabular}[]{|c|c|c|c|c|}
\hline
s110 & 1 & 3 & 4 & 6\\
\hline
3 &      1 & $\infty$ &      0 &      1\\
\hline
4 &      0 &      0 & $\infty$ &      0\\
\hline
5 & $\infty$ &      0 &      3 &      2\\
\hline
6 & $\infty$ &      1 &      0 & $\infty$\\
\hline
\end{tabular}
}\captionof*{table}{b110 = b11 + 0 + 3 = 33}
\end{table}


\begin{table}[ht]
\subfloat[][]{\begin{tabular}[]{|c|c|c|c|c}
\hline
s111 & 3 & 4 & 6 & min\\
\hline
3 & $\infty$ &      0 &      1 & 0\\
\hline
4 &      0 & $\infty$ &      0 & 0\\
\hline
5 &      0 &      3 & \cellcolor{yellow}$\infty$ & 0\\
\hline
min &      0 &      0 &      0\\
\end{tabular}
}\hfill
\subfloat[][]{\begin{tabular}[]{|c|c|c|c|}
\hline
s111 & 3 & 4 & 6\\
\hline
3 & $\infty$ &      0 &      1\\
\hline
4 &      0 & $\infty$ &      0\\
\hline
5 &      0 &      3 & $\infty$\\
\hline
\end{tabular}
}\captionof*{table}{b111 = b11 + 0 + 0 = 30}
\end{table}

\newpage


Определим дугу ветвления для разбиения множества  s01\\
\begin{flushleft}\begin{tabular}[]{|c|c|c|c|c|c|}
\hline
s01 & 2 & 3 & 4 & 5 & 6\\
\hline
1 & $\infty$ &      4 &      3 &      1 &      0\\
\hline
3 &      5 & $\infty$ &      0 &      0 &      1\\
\hline
4 &      4 &      0 & $\infty$ &      3 &      0\\
\hline
5 &      0 &      0 &      3 & $\infty$ &      2\\
\hline
6 &      1 &      1 &      0 &      2 & $\infty$\\
\hline
\end{tabular}
\captionof*{table}{(1,6)}
\end{flushleft}


\begin{table}[ht]
\subfloat[][]{\begin{tabular}[]{|c|c|c|c|c|c|c}
\hline
s010 & 2 & 3 & 4 & 5 & 6 & min\\
\hline
1 & $\infty$ &      4 &      3 &      1 & \cellcolor{yellow}$\infty$ & 1\\
\hline
3 &      5 & $\infty$ &      0 &      0 &      1 & 0\\
\hline
4 &      4 &      0 & $\infty$ &      3 &      0 & 0\\
\hline
5 &      0 &      0 &      3 & $\infty$ &      2 & 0\\
\hline
6 &      1 &      1 &      0 &      2 & $\infty$ & 0\\
\hline
min &      0 &      0 &      0 &      0 &      0\\
\end{tabular}
}\hfill
\subfloat[][]{\begin{tabular}[]{|c|c|c|c|c|c|}
\hline
s010 & 2 & 3 & 4 & 5 & 6\\
\hline
1 & $\infty$ &      3 &      2 &      0 & $\infty$\\
\hline
3 &      5 & $\infty$ &      0 &      0 &      1\\
\hline
4 &      4 &      0 & $\infty$ &      3 &      0\\
\hline
5 &      0 &      0 &      3 & $\infty$ &      2\\
\hline
6 &      1 &      1 &      0 &      2 & $\infty$\\
\hline
\end{tabular}
}\captionof*{table}{b010 = b01 + 1 + 0 = 31}
\end{table}


\begin{table}[ht]
\subfloat[][]{\begin{tabular}[]{|c|c|c|c|c|c}
\hline
s011 & 2 & 3 & 4 & 5 & min\\
\hline
3 &      5 & $\infty$ &      0 &      0 & 0\\
\hline
4 &      4 &      0 & $\infty$ &      3 & 0\\
\hline
5 &      0 &      0 &      3 & $\infty$ & 0\\
\hline
6 & \cellcolor{yellow}$\infty$ &      1 &      0 &      2 & 0\\
\hline
min &      0 &      0 &      0 &      0\\
\end{tabular}
}\hfill
\subfloat[][]{\begin{tabular}[]{|c|c|c|c|c|}
\hline
s011 & 2 & 3 & 4 & 5\\
\hline
3 &      5 & $\infty$ &      0 &      0\\
\hline
4 &      4 &      0 & $\infty$ &      3\\
\hline
5 &      0 &      0 &      3 & $\infty$\\
\hline
6 & $\infty$ &      1 &      0 &      2\\
\hline
\end{tabular}
}\captionof*{table}{b011 = b01 + 0 + 0 = 30}
\end{table}

\newpage


Определим дугу ветвления для разбиения множества  s111\\
\begin{flushleft}\begin{tabular}[]{|c|c|c|c|}
\hline
s111 & 3 & 4 & 6\\
\hline
3 & $\infty$ &      0 &      1\\
\hline
4 &      0 & $\infty$ &      0\\
\hline
5 &      0 &      3 & $\infty$\\
\hline
\end{tabular}
\captionof*{table}{(3,4)}
\end{flushleft}


\begin{table}[ht]
\subfloat[][]{\begin{tabular}[]{|c|c|c|c|c}
\hline
s1110 & 3 & 4 & 6 & min\\
\hline
3 & $\infty$ & \cellcolor{yellow}$\infty$ &      1 & 1\\
\hline
4 &      0 & $\infty$ &      0 & 0\\
\hline
5 &      0 &      3 & $\infty$ & 0\\
\hline
min &      0 &      3 &      0\\
\end{tabular}
}\hfill
\subfloat[][]{\begin{tabular}[]{|c|c|c|c|}
\hline
s1110 & 3 & 4 & 6\\
\hline
3 & $\infty$ & $\infty$ &      0\\
\hline
4 &      0 & $\infty$ &      0\\
\hline
5 &      0 &      0 & $\infty$\\
\hline
\end{tabular}
}\captionof*{table}{b1110 = b111 + 1 + 3 = 34}
\end{table}


\begin{tabular}[]{|c|c|c|}
\hline
s1111 & 3 & 6\\
\hline
4 & $\infty$ &      0\\
\hline
5 &      0 & $\infty$\\
\hline
\end{tabular}
\captionof*{table}{ b1111 = b111 + 0 + 0 = 30\\ 
$v = \{ 1,2,5,3,4,6,1 \}$
}


\newpage


Определим дугу ветвления для разбиения множества  s011\\
\begin{flushleft}\begin{tabular}[]{|c|c|c|c|c|}
\hline
s011 & 2 & 3 & 4 & 5\\
\hline
3 &      5 & $\infty$ &      0 &      0\\
\hline
4 &      4 &      0 & $\infty$ &      3\\
\hline
5 &      0 &      0 &      3 & $\infty$\\
\hline
6 & $\infty$ &      1 &      0 &      2\\
\hline
\end{tabular}
\captionof*{table}{(5,2)}
\end{flushleft}


\begin{table}[ht]
\subfloat[][]{\begin{tabular}[]{|c|c|c|c|c|c}
\hline
s0110 & 2 & 3 & 4 & 5 & min\\
\hline
3 &      5 & $\infty$ &      0 &      0 & 0\\
\hline
4 &      4 &      0 & $\infty$ &      3 & 0\\
\hline
5 & \cellcolor{yellow}$\infty$ &      0 &      3 & $\infty$ & 0\\
\hline
6 & $\infty$ &      1 &      0 &      2 & 0\\
\hline
min &      4 &      0 &      0 &      0\\
\end{tabular}
}\hfill
\subfloat[][]{\begin{tabular}[]{|c|c|c|c|c|}
\hline
s0110 & 2 & 3 & 4 & 5\\
\hline
3 &      1 & $\infty$ &      0 &      0\\
\hline
4 &      0 &      0 & $\infty$ &      3\\
\hline
5 & $\infty$ &      0 &      3 & $\infty$\\
\hline
6 & $\infty$ &      1 &      0 &      2\\
\hline
\end{tabular}
}\captionof*{table}{b0110 = b011 + 0 + 4 = 34}
\end{table}


\begin{table}[ht]
\subfloat[][]{\begin{tabular}[]{|c|c|c|c|c}
\hline
s0111 & 3 & 4 & 5 & min\\
\hline
3 & $\infty$ &      0 &      0 & 0\\
\hline
4 &      0 & $\infty$ &      3 & 0\\
\hline
6 &      1 &      0 & \cellcolor{yellow}$\infty$ & 0\\
\hline
min &      0 &      0 &      0\\
\end{tabular}
}\hfill
\subfloat[][]{\begin{tabular}[]{|c|c|c|c|}
\hline
s0111 & 3 & 4 & 5\\
\hline
3 & $\infty$ &      0 &      0\\
\hline
4 &      0 & $\infty$ &      3\\
\hline
6 &      1 &      0 & $\infty$\\
\hline
\end{tabular}
}\captionof*{table}{b0111 = b011 + 0 + 0 = 30}
\end{table}

\newpage


Определим дугу ветвления для разбиения множества  s0111\\
\begin{flushleft}\begin{tabular}[]{|c|c|c|c|}
\hline
s0111 & 3 & 4 & 5\\
\hline
3 & $\infty$ &      0 &      0\\
\hline
4 &      0 & $\infty$ &      3\\
\hline
6 &      1 &      0 & $\infty$\\
\hline
\end{tabular}
\captionof*{table}{(3,5)}
\end{flushleft}


\begin{table}[ht]
\subfloat[][]{\begin{tabular}[]{|c|c|c|c|c}
\hline
s01110 & 3 & 4 & 5 & min\\
\hline
3 & $\infty$ &      0 & \cellcolor{yellow}$\infty$ & 0\\
\hline
4 &      0 & $\infty$ &      3 & 0\\
\hline
6 &      1 &      0 & $\infty$ & 0\\
\hline
min &      0 &      0 &      3\\
\end{tabular}
}\hfill
\subfloat[][]{\begin{tabular}[]{|c|c|c|c|}
\hline
s01110 & 3 & 4 & 5\\
\hline
3 & $\infty$ &      0 & $\infty$\\
\hline
4 &      0 & $\infty$ &      0\\
\hline
6 &      1 &      0 & $\infty$\\
\hline
\end{tabular}
}\captionof*{table}{b01110 = b0111 + 0 + 3 = 33}
\end{table}


\begin{tabular}[]{|c|c|c|}
\hline
s01111 & 3 & 4\\
\hline
4 &      0 & $\infty$\\
\hline
6 & $\infty$ &      0\\
\hline
\end{tabular}
\captionof*{table}{b01111 = b0111 + 0 + 0 = 30\\ 
$v = \{ 1,6,4,3,5,2,1 \}$
}

\newpage


\begin{figure}[!h]
	\centering
	\includegraphics[scale=0.7]{var_002}
\end{figure}

\bigskip

{\large
Ответ: множество кодов всех оптимальных гамильтоновых циклов являющихся решением задачи коммивояжера на графе G есть $\{1643521, 1253461 \}$. Вес $f_o$ оптимального гамильтонова цикла равен 30.
}

\newpage


4. \quad Найдем приближенное решение задачи коммивояжера $v_1$ с помощью первого алгоритма Кристофидеса. Используя весовую матрицу смежности D графа G , построим кратчайшее связывающее дерево T с помощю алгоритма Прима.

\begin{flushleft}
\begin{tabular}[]{|c|c|c|c|c|c|c|}
\hline
$\infty$ &      0 &      6 &      5 &      3 &      2\\
\hline
	  0 & $\infty$ &      5 &      4 &      0 &      1\\
\hline
      6 &      5 & $\infty$ &      0 &      0 &      1\\
\hline
      5 &      4 &      0 & $\infty$ &      3 &      0\\
\hline
      3 &      0 &      0 &      3 & $\infty$ &      2\\
\hline
      2 &      1 &      1 &      0 &      2 & $\infty$\\
\hline
\end{tabular}
\end{flushleft}




\end{document}
